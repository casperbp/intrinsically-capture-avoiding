\section{Introduction}

% As argued by Accattoli~\cite{Accattoli19}, ``the $\lambda$-calculus is as old as computer science''.
% Still it remains a relevant subject of study thanks to its far-reaching applications in both theory and practice.
% For the same reason, the $\lambda$-calculus remains an important part of computer science curricula around the world.

A key stumbling block when learning or implementing an interpreter for the $\lambda$-calculus, is \emph{capture avoiding substitution}.
The issue is illustrated by the following term:
\begin{equation}
  (\lambda f .\, \lambda y .\, (f\ 1) + y)\ (\lambda z .\, \underbrace{y}_{\mathclap{\text{free variable}}})\ 2
  \label{eq0}
\end{equation}
This term is \emph{not} normalizable to a number value because $y$ is a \emph{free variable}; i.e., it is not bound by an enclosing $\lambda$ term.
However, using a na\"{i}ve, non capture avoiding substitution strategy to normalize the term would cause $f$ to be substituted to yield a (wrong) intermediate reduct $(\lambda y .\, ((\lambda z .\, {\color{red} y})\ 1) + y)\ 2$ where the {\color{red} red $y$} is \emph{captured}; that is, it is no longer a free variable.

Following, e.g., Curry~and~Feys~\cite{curry1958combinatory}, Plotkin~\cite{Plotkin75}, or Barendregt~\cite{DBLP:books/daglib/0067558}, the common technique to avoid such name capture is to rename variables during substitution.
For example, by renaming the $\lambda$ bound variable $y$ to $r$, we can correctly reduce term (\ref{eq0}) to $(\lambda r .\, ((\lambda z .\, y)\ 1) + r)\ 2$.
However, this renaming based substitution strategy is problematic for two reasons.
The first reason is that renaming variables give rise to intermediate reducts whose names differ from the surface program.
For applications where intermediate reducts are user facing (e.g., in error messages, or in systems based on rewriting) this gives rise to confusion.
The second problem is that implementing substitution functions that do such renaming is fiddly.
For these reasons, and since the need for renaming is only relevant for terms that contain free variables, many educational texts and research papers only define substitution for \emph{closed} terms; i.e., terms that do not contain free variables.
However, for some applications it is useful to reduce $\lambda$ terms with free variables; for example, when implementing dependent type checkers~\cite{Pareto1995ALF}, or specifying models using the mCRL2 specification language~\cite{DBLP:books/mit/GrooteM2014}.

There exist alternative techniques that can be used to define (lazy) capture avoiding substitution, such as \emph{closures}~\cite{Landin64}, \emph{de Bruijn indices}~\cite{de_Bruijn_1972}, \emph{explicit substitutions}~\cite{AbadiCCL91}, and \emph{locally nameless}~\cite{Chargueraud12}.
However, traditional naive substitution is sometimes preferred because intermediate reducts are easy to inspect and compare.
% They may also be preferable for teaching the $\lambda$-calculus because the aforementioned techniques are essentially clever encodings of naive substitution.

In recent years, Eelco Visser and I were teaching the $\lambda$-calculus to undergraduate students by having them implement definitional interpreters and substitution functions.
To this end, we used a technique for substitution in $\lambda$ terms that is capture avoiding but does not require renaming of bound variables during substitution.
The idea is to delimit and distinguish those terms in abstract syntax trees (ASTs) that have already been computed to normal forms, and to never substitute inside those.
For example, using $\lfloor$ and $\rfloor$ for this delimiter, an intermediate reduct of the term labeled (\ref{eq0}) above is $(\lambda y.\, (\colorbox{gray!30}{$\lfloor (\lambda z.\, y) \rfloor$}\ 1) + y)\ 2$.
Here the delimited \colorbox{gray!30}{highlighted} term is closed under substitution, such that the substitution of $y$ for $2$ is not propagated past the delimiter; i.e., using $\leadsto$ to denote reduction:
\begin{align*}
  & (\lambda f .\, \lambda y .\, (f\ 1) + y)\ (\lambda z .\, y)\ 2
  \\
  \leadsto\quad & (\lambda y.\, (\colorbox{gray!30}{$\lfloor (\lambda z.\, y) \rfloor$}\ 1) + y)\ 2
  \\
  \leadsto\quad & (\colorbox{gray!30}{$\lfloor (\lambda z.\, y) \rfloor$}\ 1) + 2
  \\
  \leadsto\quad & ((\lambda z.\, y) 1) + 2
  \\
  \leadsto\quad & y + 2
\end{align*}
%
These reductions are equivalent to using a renaming based substitution function.
However, our renamingless substitution strategy does not rename variables, and (as we demonstrate in \cref{sec:interpreting-open}) is about as simple to define and implement as a substitution for closed terms.

The idea of distinguishing values from plain terms is not new (for example, \emph{reduction semantics}~\cite{FelleisenH92} commonly make this distinction), but I am not aware of this idea being applied to implement capture avoiding substitution functions outside of our course.\footnote{This judgement is made solely by the author of the present paper. Eelco and I never discussed the novelty of the technique.}
This paper presents the technique, and shows that the resulting substitution functions are capture avoiding.
We make the following technical contributions:
%
\begin{itemize}
\item We present a technique (\cref{sec:interpretation}) for renamingless capture avoiding substitution, that is about as simple to understand and implement as substitution for closed terms.
\item We present an intrinsically scoped capture avoiding normalizer (\cref{sec:normalization}) for the untyped $\lambda$-calculus, which explains how and why our technique is capture avoiding.
\end{itemize}
%
The paper is a literate Agda paper whose sources can be found here: \url{https://github.com/casperbp/intrinsically-capture-avoiding}


%%% Local Variables:
%%% reftex-default-bibliography: ("../references.bib")
%%% End:
