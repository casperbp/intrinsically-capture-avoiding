\documentclass[a4paper,UKenglish,cleveref, autoref, thm-restate]{oasics-v2021}

%for A4 paper format use option "a4paper", for US-letter use option "letterpaper"
%for british hyphenation rules use option "UKenglish", for american hyphenation rules use option "USenglish"
%for section-numbered lemmas etc., use "numberwithinsect"
%for enabling cleveref support, use "cleveref"
%for enabling autoref support, use "autoref"
%for anonymousing the authors (e.g. for double-blind review), add "anonymous"
%for enabling thm-restate support, use "thm-restate"
%for enabling a two-column layout for the author/affilation part (only applicable for > 6 authors), use "authorcolumns"
%for producing a PDF according the PDF/A standard, add "pdfa"

\usepackage{mathtools}
\usepackage{stmaryrd}
\usepackage{wasysym}
\usepackage{agda}
\usepackage{agdadimmed}
\usepackage{newunicodechar}

%\pdfoutput=1 %uncomment to ensure pdflatex processing (mandatatory e.g. to submit to arXiv)
%\hideOASIcs %uncomment to remove references to OASIcs series (logo, DOI, ...), e.g. when preparing a pre-final version to be uploaded to arXiv or another public repository

\bibliographystyle{plainurl}% the mandatory bibstyle

\makeatletter
\newcommand{\crefnames}[3]{%
  \@for\next:=#1\do{%
    \expandafter\crefname\expandafter{\next}{#2}{#3}%
  }%
}
\makeatother

\crefnames{section}{\S}{\S\S}

%%
%% Agda typesetting commands shorthands, for
%% manual typesetting of inline code
%%

\newcommand{\af}{\AgdaFunction}
\newcommand{\un}{\AgdaUnderscore}
\newcommand{\ad}{\AgdaDatatype}
\newcommand{\ab}{\AgdaBound}
\newcommand{\ac}{\AgdaInductiveConstructor}
\newcommand{\aF}{\AgdaField}
\newcommand{\as}{\AgdaSymbol}
\newcommand{\ak}{\AgdaKeyword}
\newcommand{\ap}{\AgdaPrimitiveType}
\newcommand{\an}{\AgdaNumber}
\newcommand{\aC}{\AgdaComment}
\newcommand{\am}{\AgdaModule}

\setlength{\parskip}{0em} 
\setlength\mathindent{0.2cm}

% \newcommand{\citep}[1]{\cite{#1}}
% \newcommand{\citet}[1]{\citeauthor{#1}~\cite{#1}}


%%
%% Unicode for typesetting Agda code
%%%%%%%%%%%%%%%%%%%
%% AGDA UNICODE %%%
%%%%%%%%%%%%%%%%%%%

% hacks

\newunicodechar{◂}{} % for repeating defs from stdlib for example

%%
%% Symbols
%%

\newcommand\superequiv{\mathrel{\rlap{\raisebox{\fontdimen22\textfont2}{$=$}}\raisebox{-0.5\fontdimen22\textfont2}{$ = $}}}

\newunicodechar{≈}{\ensuremath{\mathnormal\approx}}
\newunicodechar{×}{\ensuremath{\mathnormal\times}}
\newunicodechar{→}{\ensuremath{\mathnormal\to}}
\newunicodechar{←}{\ensuremath{\mathnormal\leftarrow}}
\newunicodechar{⟦}{\ensuremath{\mathnormal\llbracket}}
\newunicodechar{⟧}{\ensuremath{\mathnormal\rrbracket}}
\newunicodechar{ℕ}{\ensuremath{\mathbb{N}}}
\newunicodechar{⊕}{\ensuremath{\mathnormal\oplus}}
\newunicodechar{∔}{\ensuremath{\mathnormal\dotplus}}
\newunicodechar{⋎}{\ensuremath{\mathnormal\curlyvee}}
\newunicodechar{⊞}{\ensuremath{\mathnormal\boxplus}}
\newunicodechar{⇒}{\ensuremath{\mathnormal\Rightarrow}}
\newunicodechar{⟨}{\ensuremath{\mathnormal\langle}}
\newunicodechar{⟩}{\ensuremath{\mathnormal\rangle}}
\newunicodechar{∪}{\ensuremath{\mathnormal\cup}}
\newunicodechar{⦃}{\ensuremath{\mathnormal\{\mskip-4.5mu\mid}}
\newunicodechar{⦄}{\ensuremath{\mathnormal\mid\mskip-4.5mu\}}}
\newunicodechar{⊎}{\ensuremath{\mathnormal\uplus}}
% \newunicodechar{∗}{\ensuremath{\mathnormal\ast}}
\newunicodechar{✴}{\ensuremath{\mathnormal\ast}}
\newunicodechar{↦}{\ensuremath{\mathnormal\mapsto}}
\newunicodechar{≡}{\ensuremath{\mathnormal\equiv}}
\newunicodechar{∀}{\ensuremath{\mathnormal\forall}}
\newunicodechar{∙}{\ensuremath{\mathnormal\bullet}}
\newunicodechar{≣}{\ensuremath{\mathnormal\superequiv}}
\newunicodechar{▿}{\ensuremath{\mathnormal\triangledown}}
\newunicodechar{▸}{\raisebox{0.25ex}{\scaleobj{0.65}{\ensuremath{\mathnormal\blacktriangleright}}}}
\newunicodechar{∼}{\ensuremath{\mathnormal\sim}}
\newunicodechar{≤}{\ensuremath{\mathnormal\leq}}
\newunicodechar{↔}{\ensuremath{\mathnormal\leftrightarrow}}
\newunicodechar{⊂}{\ensuremath{\mathnormal\subset}}
\newunicodechar{∘}{\ensuremath{\mathnormal\circ}}
\newunicodechar{∃}{\ensuremath{\mathnormal\exists}}
\newunicodechar{↓}{\ensuremath{\mathnormal\downarrow}}
\newunicodechar{↑}{\ensuremath{\mathnormal\uparrow}}
\newunicodechar{⊔}{\ensuremath{\mathnormal\sqcup}}
\newunicodechar{⊢}{\ensuremath{\mathnormal\vdash}}
\newunicodechar{◇}{\ensuremath{\mathnormal\diamond}}
\newunicodechar{⊙}{\ensuremath{\mathnormal\odot}}
\newunicodechar{⊤}{\ensuremath{\mathnormal\top}}
\newunicodechar{⊥}{\ensuremath{\mathnormal\bot}}
\newunicodechar{∣}{\ensuremath{\mathnormal\mid}}
\newunicodechar{‵}{\ensuremath{^\backprime}}
\newunicodechar{′}{\ensuremath{^\prime}}
\newunicodechar{″}{\ensuremath{^{\prime\prime}}}
\newunicodechar{‴}{\ensuremath{^{\prime\prime}}}
\newunicodechar{∅}{\ensuremath{\emptyset}}
\newunicodechar{≺}{\ensuremath{\mathnormal{\prec}}}
\newunicodechar{≼}{\ensuremath{\mathnormal{\preceq}}}
\newunicodechar{∩}{\ensuremath{\mathnormal{\cap}}}
\newunicodechar{⟪}{\ensuremath{\langle\kern-.2em\langle}}
\newunicodechar{⟫}{\ensuremath{\rangle\kern-.2em\rangle}}
\newunicodechar{⊆}{\ensuremath{\mathnormal{\subseteq}}}
\newunicodechar{⊇}{\ensuremath{\mathnormal{\supseteq}}}
\newunicodechar{▻}{\ensuremath{\mathnormal\vartriangleright}}
\newunicodechar{∷}{\ensuremath{::}}
\newunicodechar{►}{\ensuremath{\mathnormal{\blacktriangleright}}}
\newunicodechar{▹}{\ensuremath{\mathnormal{\triangleright}}}
\newunicodechar{□}{\ensuremath{\mathnormal{\square}}}
\newunicodechar{⋯}{\ensuremath{\mathnormal{\cdots}}}
\newunicodechar{▣}{\ensuremath{\mathnormal{\ldots}}}
\newunicodechar{⋮}{\ensuremath{\mathnormal{\quad\quad\vdots}}}
\newunicodechar{∈}{\ensuremath{\mathnormal{\in}}}
\newunicodechar{⊑}{\ensuremath{\mathbin{\sqsubseteq}}}
\newunicodechar{𝓑}{\ensuremath{\mathnormal{\gg\!\!=}}}
\newunicodechar{𝓒}{\ensuremath{\mathbf{\mathscr{C}}}}
\newunicodechar{𝓓}{\ensuremath{\mathbf{\mathscr{D}}}}
\newunicodechar{♯}{\ensuremath{\sharp}}
\newunicodechar{∎}{\ensuremath{\qed}}
\newunicodechar{↣}{\ensuremath{\rightarrowtail}}
\newunicodechar{⟶}{\ensuremath{\longrightarrow}}

%%
%% Greek
%%

\newunicodechar{φ}{\ensuremath{\phi}}
\newunicodechar{Φ}{\ensuremath{\Phi}}
\newunicodechar{ψ}{\ensuremath{\psi}}
\newunicodechar{μ}{\ensuremath{\mu}}
\newunicodechar{ρ}{\ensuremath{\rho}}
\newunicodechar{σ}{\ensuremath{\sigma}}
\newunicodechar{ξ}{\ensuremath{\xi}}
\newunicodechar{Ξ}{\ensuremath{\Xi}}
\newunicodechar{λ}{\ensuremath{\lambda}}
\newunicodechar{ε}{\ensuremath{\epsilon}}
\newunicodechar{γ}{\ensuremath{\gamma}}
\newunicodechar{Σ}{\ensuremath{\Sigma}}
\newunicodechar{Δ}{\ensuremath{\Delta}}
\newunicodechar{Π}{\ensuremath{\Pi}}
\newunicodechar{Γ}{\ensuremath{\Gamma}}
\newunicodechar{η}{\ensuremath{\eta}}
\newunicodechar{ζ}{\ensuremath{\zeta}}


%%
%% subscript/superscript
%%

\newunicodechar{₀}{$_{0}$}
\newunicodechar{₁}{$_{1}$}
\newunicodechar{₂}{$_{2}$}
\newunicodechar{₃}{$_{3}$}
\newunicodechar{₄}{$_{4}$}
\newunicodechar{₅}{$_{5}$}
\newunicodechar{₆}{$_{6}$}
\newunicodechar{₇}{$_{7}$}
\newunicodechar{₈}{$_{8}$}
\newunicodechar{₉}{$_{9}$}
\newunicodechar{¹}{$^{1}$}
\newunicodechar{⁻}{$^{-}$}
\newunicodechar{ᴬ}{$^{A}$}
\newunicodechar{ᴮ}{$^{B}$}
\newunicodechar{ᴴ}{$^{H}$}
\newunicodechar{ˣ}{$^{×}$}
\newunicodechar{ᵈ}{$^{d}$}
\newunicodechar{ᵘ}{$^{u}$}
\newunicodechar{ᶠ}{$^{F}$}
\newunicodechar{ⁱ}{$^{i}$}
\newunicodechar{ᵒ}{$^{o}$}
\newunicodechar{ˢ}{$^{s}$}
\newunicodechar{ˡ}{$^{l}$}
\newunicodechar{ʳ}{$^r$}
\newunicodechar{ᴰ}{$^D$}
\newunicodechar{ᵇ}{$^{b}$}
\newunicodechar{ᵐ}{$_m$}
\newunicodechar{ⁿ}{$_n$}
\newunicodechar{ₙ}{$_n$}
\newunicodechar{ₘ}{$_m$}
\newunicodechar{ᵛ}{$_v$}
\newunicodechar{ᵍ}{$_f$}
\newunicodechar{ᵢ}{$_i$}
\newunicodechar{ₗ}{$_l$}
\newunicodechar{ᵣ}{$_r$}
\newunicodechar{ₛ}{$_s$}
\newunicodechar{ₜ}{$_t$}
\newunicodechar{ᶜ}{$^c$}
\newunicodechar{ₐ}{$_a$}
\newunicodechar{∶}{$:$}
\newunicodechar{̅}{$^{\textit{d}}$}

\newunicodechar{𝑡}{\textit{t}}
\newunicodechar{ℎ}{\textit{h}}
\newunicodechar{𝑟}{\textit{r}}
\newunicodechar{𝑜}{\textit{o}}
\newunicodechar{𝑤}{\textit{w}}
\newunicodechar{𝑐}{\textit{c}}
\newunicodechar{𝑎}{\textit{a}}
\newunicodechar{𝑡}{\textit{t}}



%% Title information
\title{Renamingless Capture-Avoiding Substitution, Intrinsically Scoped}

%% Author with single affiliation.
\author{Casper {Bach Poulsen}}{Delft University of Technology, Netherlands \and \url{http://www.casperbp.net} }{c.b.poulsen@tudelft.nl}{https://orcid.org/0000-0003-0622-7639}{}%TODO mandatory, please use full name; only 1 author per \author macro; first two parameters are mandatory, other parameters can be empty. Please provide at least the name of the affiliation and the country. The full address is optional. Use additional curly braces to indicate the correct name splitting when the last name consists of multiple name parts.
%
\authorrunning{C. Bach Poulsen}
% First names are abbreviated in the running head.
% If there are more than two authors, 'et al.' is used.

\Copyright{Jane Open Access and Joan R. Public} %TODO mandatory, please use full first names. LIPIcs license is "CC-BY";  http://creativecommons.org/licenses/by/3.0/

\begin{CCSXML}
<ccs2012>
   <concept>
       <concept_id>10011007.10011006.10011039.10011311</concept_id>
       <concept_desc>Software and its engineering~Semantics</concept_desc>
       <concept_significance>500</concept_significance>
       </concept>
   <concept>
       <concept_id>10003752.10003790.10002990</concept_id>
       <concept_desc>Theory of computation~Logic and verification</concept_desc>
       <concept_significance>300</concept_significance>
       </concept>
 </ccs2012>
\end{CCSXML}

\ccsdesc[500]{Software and its engineering~Semantics}
\ccsdesc[300]{Theory of computation~Logic and verification}

\keywords{Capture-avoiding substitution, Untyped lambda calculus, Agda, Dependent types} %TODO mandatory; please add comma-separated list of keywords

%\category{} %optional, e.g. invited paper

%\relatedversion{} %optional, e.g. full version hosted on arXiv, HAL, or other respository/website
%\relatedversiondetails[linktext={opt. text shown instead of the URL}, cite=DBLP:books/mk/GrayR93]{Classification (e.g. Full Version, Extended Version, Previous Version}{URL to related version} %linktext and cite are optional

%\supplement{}%optional, e.g. related research data, source code, ... hosted on a repository like zenodo, figshare, GitHub, ...
%\supplementdetails[linktext={opt. text shown instead of the URL}, cite=DBLP:books/mk/GrayR93, subcategory={Description, Subcategory}, swhid={Software Heritage Identifier}]{General Classification (e.g. Software, Dataset, Model, ...)}{URL to related version} %linktext, cite, and subcategory are optional

%\funding{(Optional) general funding statement \dots}%optional, to capture a funding statement, which applies to all authors. Please enter author specific funding statements as fifth argument of the \author macro.

%\acknowledgements{I want to thank \dots}%optional

%\nolinenumbers %uncomment to disable line numbering

%Editor-only macros:: begin (do not touch as author)%%%%%%%%%%%%%%%%%%%%%%%%%%%%%%%%%%
\EventEditors{John Q. Open and Joan R. Access}
\EventNoEds{2}
\EventLongTitle{42nd Conference on Very Important Topics (CVIT 2016)}
\EventShortTitle{CVIT 2016}
\EventAcronym{CVIT}
\EventYear{2016}
\EventDate{December 24--27, 2016}
\EventLocation{Little Whinging, United Kingdom}
\EventLogo{}
\SeriesVolume{42}
\ArticleNo{23}
%%%%%%%%%%%%%%%%%%%%%%%%%%%%%%%%%%%%%%%%%%%%%%%%%%%%%%

\begin{document}

\maketitle

\begin{abstract}
  We describe a simple and direct technique for capture avoiding substitution of untyped $\lambda$ terms that avoids the need to rename bound variables during substitution.
  We demonstrate how this substitution technique yields correct normalization of open $\lambda$ terms to weak head normal form.
  We also give an intrinsically scoped syntax for untyped $\lambda$ terms.
  Using this syntax we show that the substitution technique is, indeed, capture avoiding.
\end{abstract}

\section{Introduction}

As argued by \citet{Accattoli19}, ``the $\lambda$-calculus is as old as computer science''; still, it remains relevant subject of study due to its far-reaching applications in both theory and practice.
For the same reason, the (untyped) $\lambda$-calculus is commonly taught in (under)graduate computer science programs at universities around the world.

A key stumbling block of learning and implementing the $\lambda$-calculus, is \emph{capture avoiding substitution}.
The issue is illustrated by the following term:
\begin{equation}
  (\lambda f .\, \lambda y .\, (f\ 1) + y)\ (\lambda z .\, \underbrace{y}_{\mathclap{\text{occurs free}}})\ 2
  \label{eq0}
\end{equation}
This term is \emph{not} reducible to a number value since the underbraced $y$ is a free variable; i.e., a variable that is not bound by an enclosing $\lambda$ term.
However, a na\"{i}ve, non capture avoiding substitution strategy could cause $f$ to be wrongly substituted to yield a reduct where $y$ is captured, such as $(\lambda y .\, ((\lambda z .\, {\color{red}{y}})\ 1) + y)\ 2$ which \emph{is} reducible to $4$.

Following, e.g., \citet{curry1958combinatory}, \citet{Plotkin75}, or \citet{DBLP:books/daglib/0067558}, the common technique to avoid such name capture is to rename variables during substitution.
Using renaming, the substitution of $f$ in \cref{eq0} instead yields the correct reduct $(\lambda r .\, ((\lambda z .\, y)\ 1) + r)\ 2$.

However, implementing renaming in a \emph{definitional interpreter}~\citep{reynolds98definitional} is a delicate and error-prone matter.
For this reason, and since the need for renaming is only relevant for terms that contain \emph{free variables}, educational texts---and many research papers---often define substitution only for \emph{closed} terms; i.e., terms that do not contain free variables.
But for some applications, such as implementing dependent type checking~\citep{Pareto1995ALF}, it can be desirable to reduce $\lambda$ terms involving free variables.

There exists alternative techniques that can be used to define (lazy) capture avoiding substitution, such as \emph{closures and environments}~\citep{Landin64}, \emph{de Bruijn indices}~\citep{de_Bruijn_1972}, and \emph{explicit substitutions}~\citep{AbadiCCL91}.
But naive and eager substitution functions are usually for teaching the $\lambda$-calculus, because they provide an intuitive framework for understanding the concept that the aforementioned techniques encode.

In recent years, Eelco Visser and I have been teaching the $\lambda$-calculus to bachelor students by having them implement definitional interpreters using a technique for capture avoiding substitution that avoids the need to rename bound variables.
The idea is to distinguish those terms in abstract syntax trees (ASTs) that have already been computed to normal forms, and to never substitute inside of those terms.
The benefit of this approach is that it makes it as simple to understand and implement substitutions for \emph{open} terms (i.e., terms that may contain free variables) as it is for closed terms.

The idea of distinguishing values from plain terms is not new (for example, \emph{reduction semantics}~\citep{FelleisenH92} commonly make this distinction), but I am not aware of this idea being applied to implement capture avoiding substitution functions outside of our course.\footnote{This judgement is made solely by the author of the present paper. I never had the chance to discuss the novelty of the technique with Eelco, in spite of us coming up with the idea of using the technique in the course together on Eelco's whiteboard.}
This paper presents the technique, and proves that the resulting substitution functions are capture avoiding.
We make the following technical contributions:

\begin{itemize}
\item We present a technique (\cref{sec:interpretation}) for capture avoiding substitution that does not require renaming, and that is about as simple to understand and implement as substitution functions for closed terms.
\item We present an intrinsically capture avoiding definitional interpreter (\cref{sec:normalization}) for the untyped $\lambda$-calculus which proves that the technique is, indeed, capture avoiding.
\end{itemize}





%%% Local Variables:
%%% reftex-default-bibliography: ("../references.bib")
%%% End:

\input{sections/2-interpretation.tex}
\input{sections/3-normalization.tex}
\section{Conclusion and Future Directions}
\label{sec:conclusion}

We have presented a technique for capture avoiding substitution that does not require renaming of bound variables.
The technique results in substitution functions that perform capture avoiding substitution involving open terms, but which is as simple to implement and understand as substitution functions involving only closed terms.
This makes this style of substitution attractive for, e.g., teaching and learning the untyped $\lambda$ calculus.
By intrinsically typing substitution functions we have shown that our technique is indeed capture avoiding.

This paper only considers normalization to weak head normal form, using a \emph{call-by-value} normalization strategy.
We conjecture that the techniques are equally applicable to \emph{call-by-name normalization} strategies, as well as normalization to stronger normal forms.
We leave verification of this conjecture as future work.




%% Bibliography
\bibliography{references}



%% Appendix
%% \appendix
%% \section{Appendix}
%% 
%% Text of appendix \ldots
%% 
\end{document}
